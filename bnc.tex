\newcommand{\myteam}{Project 9} % Name of your team goes here.
\title{D-regular graph(Directed | Undirected)\footnote{This is a report on the course 
    project for the course CS 213 Software System Lab}} % You may change the title if you want.
\author{\myteam\footnote{Email IDs of team members:170010009\texttt{@iitdh.ac.in},170030038\texttt{@iitdh.ac.in}}\\
    {\small Lokesh Nirania (170010009),} % Here arrange the the member's name and roll nos.  
    {\small Mandeep Bawa (170030038)}\\ % in the increasing order of roll nos.
    {\small Computer Science and Engineering, IIT Dharwad}\\% Do not change this.
}
\date{\today}

\documentclass[12pt]{article}
\usepackage{fullpage}
\usepackage{hyperref}
\begin{document}
\maketitle

\begin{abstract} % You may change the abstract if you want.
This paper describes the working of the code \textbf{ps\_graph.ps1} written in powershell script to make a \textit{d-regular graph} as a part of the course project in CS 213 Software System Lab.
\end{abstract}

\section{Working Of Code}
Our program takes input from user for undirected or directed graph. After this step it ask if user want to give \textit{matrix file}.   
\subsection{Matrix file}
\begin{itemize}
\item{Any m*n matrix file is required for this program to work.}
\item{Any char or integer(0 for no link) can be given in a cell of input matrix. These will represent edge labels in the output graph.}
\item{If matrix input is not valid (not given as a square matrix or left some cells empty) then our program filter the matrix input and give default value as 0 for empty spaces.}
\item{If user does not want to give a matrix then our program ask for the number of vertices required in output graph.}
\item{After this program execute and final pdf is displayed.}
\end{itemize}
\section{Step by Step Improvement}
Initially program for simple directed graph was made using only latex in which number of vertices were given manually in a tex file. \\
Then we tried our best to make it user friendly using powershell.\\
All the Latex syntax were written in a file using powershell.\\
Powershell takes input from user and write a file in latex.\\
After the file has been written, powershell will execute the latex file and will show the PDF. \\
It will automatically remove all the *tex,*log files after the work is done. \\

\section{Applications in real life}
\begin{enumerate}
\item{It can be used to design a \textbf{map} for road networks which can be named distincty using distinct chars and integers 0-9 in input matrix.}
\item{It can be used for building \textbf{network frameworks} and to show the permission access to various users.}
\item{Facebook use graphs to \textbf{find mutual friends} and give new friend suggestions. }
\item{In computer science it is used to represent the \textbf{flow of computation}.}
\item{\textbf{Google Maps} uses graphs for building transportation systems,where intersection of two(or more) roads are considered to be a vertex and the road connecting two vertices is considered to be an edge.}
\end{enumerate}
\quad
\quad
\begin{center}
\quad
\quad
- - - - - - - - - End of report - - - - - - - - -
\end{center}

\end{document}
