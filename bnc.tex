\newcommand{\myteam}{Project 9} % Name of your team goes here.
\title{D-regular graph(Directed | Undirected)\footnote{This is a report on the course 
    project for the course CS 213 Software System Lab}} % You may change the title if you want.
\author{\myteam\footnote{Email IDs of team members:170010009\texttt{@iitdh.ac.in},170030038\texttt{@iitdh.ac.in}}\\
    {\small Lokesh Nirania (170010009),} % Here arrange the the member's name and roll nos.  
    {\small Mandeep Bawa (170030038)}\\ % in the increasing order of roll nos.
    {\small Computer Science and Engineering, IIT Dharwad}\\% Do not change this.
}
\date{\today}

\documentclass[12pt]{article}
%\usepackage{fullpage}
\usepackage{hyperref}
\begin{document}
\maketitle

\begin{abstract} % You may change the abstract if you want.
This paper describes the techniques followed by our team to implement a powershell script to make a \textit{graph in latex} as a part of the course CS 213 Software System Lab.
\end{abstract}

\section{Working Of Code}
Our program takes input from user for undirected or directed graph. After this step it ask if user want to give \textit{matrix file}.   
\subsection{Matrix file}
A n*n matrix file is required for this program to work. Any char or integer(0 for no link) can be given in a cell of matrix. These will represent edge labels in the formed graph.
If user does not want to give a matrix then our program ask for the number of vertices required in graph.Then our program execute and final pdf is displayed.
\section{Step by Step Improvement}
Firstly program for simple directed graph was made using only latex in which we need to manually set number of vertices in a file and execute.
Then we tried our best to make it user friendly using powershell.\\
All the Latex syntax were written in a file using powershell.\\
Powershell takes input from user and write a file in latex.\\
After the file has been written, powershell will execute the latex file and will display the PDF on screeen.
It will automatically remove all the *tex,*log files after the work is done.



%A paper by Tao and Vu~\cite{TaoV17} talks about something quite different but this is how you can refer to a research paper/books etc.
%You can also refer to an online link~\cite{bworld}. 

%\bibliographystyle{abbrv}
%\bibliography{bnc}

\end{document}
This is never printed
